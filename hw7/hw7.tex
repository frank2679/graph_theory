\documentclass[12pt]{article}
\usepackage{geometry}
\usepackage{amsmath}
\usepackage{amsthm}
\usepackage{graphicx}  % used to insert graph
\usepackage{subcaption}  % used to insert subgraph
\usepackage{amssymb}
\usepackage{enumitem}
\usepackage{fancyhdr}
\usepackage{tikz}
\usetikzlibrary{trees}

%% for code listing
\usepackage{listings}
\usepackage{color}
\definecolor{dkgreen}{rgb}{0,0.6,0}
\definecolor{gray}{rgb}{0.5,0.5,0.5}
\definecolor{mauve}{rgb}{0.58,0,0.82}
\lstset{frame=tb,
  language=C,
  aboveskip=3mm,
  belowskip=2mm,
  showstringspaces=false,
  columns=flexible,
  basicstyle={\small\ttfamily},
  numbers=none,
  numberstyle=\tiny\color{gray},
  keywordstyle=\color{blue},
  commentstyle=\color{dkgreen},
  stringstyle=\color{mauve},
  breaklines=false, %true,
  breakatwhitespace=false, %true,
  tabsize=3
}


\geometry{left=2.5cm,right=3 cm, top=2.5cm,bottom=2.5cm}
\lhead{Chapter 4-1}
\chead{Graph Theory HW-7}
\rhead{Yang Hang, r03942126}

\begin{document}
\pagestyle{fancy}
\textbf{Notations:} $n(G)$: number of vertices of graph $G$.
\begin{enumerate}
\item[\textbf{4.1}] 
\begin{proof}
Let $g_n$ denote number of perfect matching of $Q_n$. \\
Assume a series $h_n = 2^{2^{n-2}}$, we have $h_n = h_{n-1}^2$. First, $g_2 = h_2$. Then to prove $g_n \geq h_n = h_{n-1}^2$, it is sufficient to prove $g_n \geq g_{n-1}^2$. \\
$Q_n$ is induced by $Q_{n-1}$. $n(Q_n) = 2n(Q_{n-1})$. $Q_n$ is formed by $Q_{n-1}$ with a more bit sequence and its counterpart $Q_{n-1}^\prime$ having the same topology with $Q_{n-1}$. Vertex in one part has 1-to-1 link with vertex of the other part. \\
In this way, $Q_{n-1}, Q_{n-1}^\prime$ both have $g_{n-1}$ perfect matching. Therefore, $g_n \geq g_{n-1}^2$. 
\end{proof}
% -----------------------------------------
\item[\textbf{4.6}] 
We can get that $G$ has at least $r$ X-perfect matchings. 
\begin{proof}
With some vertex $x$ in $X$, it has at least $r$ edges with $Y$. Select one edge $xy_1$, the remaining graph $G^\prime$ still satisfies that $N(S) \geq |S|$, i.e. a perfect matching exists in $G^\prime$. We have $r$ choices of $xy_i$. Therefore, $G$ has at least $r$ different $X$-perfect matching. 
\end{proof}
% -----------------------------------------
\item[\textbf{4.8}] 
\begin{proof}
Proof consists of two parts. The two people are denoted by $A,B$.\\ 
\textbf{(1)} If $G$ has perfect matching $M$. The strategy of $B$ is to select the vertex that the edge between this vertex and vertex selected by $A$ last time belongs to $M$.\\
Whatever dose $A$ selects, $B$ can always find an vertex adjacent that the edge between them belongs to $M$, because $G$ has a perfect matching.\\
\textbf{(2)} If $G$ doesn't have perfect matching. The strategy of $A$ is to select the vertex not joining with edge which belongs to maximum matching.\\
In this way, $B$ can only select an vertex joining an edge which doesn't belong to the maximum matching. After $B$'s turn, $A$ can always find an vertex which joining an edge belongs to the maximum matching.
\end{proof}
% -----------------------------------------
\item[\textbf{4.10}]\textbf{Birkhoff Theorem Proof} \\
Form associated graph of \textbf{doubly stochastic matrix} $Q$. Use following matrix as an example. 
\begin{equation*}
Q =
\begin{pmatrix}
1/3 & 1/2 & 1/6 \\
0   & 1/3 & 2/3 \\
2/3 & 1/6 & 1/6
\end{pmatrix}
\end{equation*}
Associated graph is as following Figure \ref{10}. 
\begin{figure}[htp!]
	\centering
	\includegraphics[width=.25\linewidth]{./figure/4_10.png}
	\caption{Associated graph of $Q$}
	\label{10}
\end{figure} 
\newtheorem{lemma}{Lemma} 
\begin{lemma}
Associated graph of doubly stochastic matrix has a perfect matching.
\end{lemma}
\begin{proof}[Proof of lemma]
There is a subset $A$ of the vertices in one part such that the set $R(A)$ of all vertices connected to some vertex in $A$ has strictly less than $|A|$ elements. Without loss of generality, $A$ is a set of vertices representing rows while $R(A)$ represents columns. \\
We have $\sum_{i\in A, j\in R(A)} x_{i,j} = |A|$
and $\sum_{i\in A, j\in R(A)} x_{i,j} \leq |R(A)|$. 
Then Hall's condition is satisfied, a perfect matching exists. 
\end{proof}
\begin{proof}[Proof of Birkhoff Theorem] 
We proceed by induction on non-zero entries in matrix. 
Let $M_0$ be doubly stochastic matrix. By the lemma, the associated graph has a perfect matching. Use above $Q$ as an example, $(1,1), (2,2),(3,3)$ is a perfect matching.
Let $\alpha_0$ be the minimum of the entries indexed by the matching, $1/6$. 
Let $P_0$ be the permutation matrix with 1 exactly at position of the matching. 
Then $M_0-\alpha_0 P_0$ has non-negative entries. Dividing each entry by $(1-\alpha_0)$ in $M_0-\alpha_0 P_0$ gives a doubly stochastic matrix $M_1$.
Then $M_0 = \alpha_0 P_0 + (1 - \alpha_0) M_1$, where $M_1$ has less non-negative entries than $M_0$.
By induction hypothesis, $M_1$ may be written as $M_1 = \alpha_1 P1 + \alpha_2 P_2 \dots + \alpha_n P_n$. 
Therefore, we have
\begin{equation*}
Q = M_0 = \alpha_0 P_1 + (1-\alpha_0)\alpha_1 P_1 + \dots +(1- \alpha_0)\alpha_n P_n
\end{equation*}
Let $\alpha_0 = c_1, (1-\alpha_0)\alpha_1 = c_2, \dots $, where  $$\sum_i^m c_i = \alpha_0 + (1-\alpha_0)(\alpha_1 + \alpha_2 + \dots + \alpha_{m-1}) = \alpha_0 + (1-\alpha_0) = 1$$
\end{proof}
% -----------------------------------------
\item[\textbf{4.13}] 
\begin{proof}[a Proof]
\textit{Necessity}. $S$ is an independent set. That means no edges joining vertices in $S$. Therefore, $\overline{S}$ is a vertex cover.\\
\textit{Sufficiency}. $\overline{S}$ is an vertex cover. If $S$ is not an independent set, i.e. edge ${e: xy}$ exists joining vertices $x,y$ in $S$. It contradicts that $\overline{S}$ covers all edges of graph.   \\
Hence, every maximum independent set is the complement of a minimum vertex cover, and $\alpha(G) + \beta(G) = |V(G)|$.
\end{proof}
\begin{proof}[b Proof]
With a maximum matching $M$, we try to construct an edge cover of size $n(G)- |M|$. Since the minimum edge cover won't be larger than this cover, it implies that $\beta^\prime(G) \leq n(G) - \alpha^\prime(G)$.
On the other hand, with a minimum edge cover $L$, we try to construct a maximum matching of size $n(G)-|L|$. Since maximum matching, $\alpha^\prime(G) \geq n(G) - \beta^\prime(G)$. Then we have $\alpha^\prime(G) + \beta^\prime(G) = n(G)$. Following are procedures to find such edge cover and mathcing.\\
Let $M$ be the maximum matching of graph $G$. We construct an edge cover by adding to $M$ one edge incident to each unsaturated vertex, one edge to one vertex. The number of unsaturated vertices is $n(G) - 2|M|$, including $M$ we will get an edge cover of size $n(G)-2|M| + |M| = n(G)-|M|$.\\
Let $L$ be the minimum edge cover. If both endpoints of edge $e$ belong to edge in $L$ other than $e$, then $e \not\in L$, since $L-{e}$ is also an edge cover. 
Hence each component of $L$ has only one vertex with degree larger than 1, i.e. forming a star. Then we make a matching by selecting one edge of each star. Size of the matching will be $n(G)- |L|$.
\end{proof}
\begin{proof}[c Proof]
From above $a,b$, we have $\alpha(G)+\beta(G)=\alpha^\prime(G)+ \beta^\prime(G)$. From \textbf{Thm 4.9}, we have $\alpha^\prime(G) = \beta(G)$. Hence, $\alpha(G) = \beta^\prime(G)$.
\end{proof}
\end{enumerate}



%@online{WinNT,
%  author = {MultiMedia LLC},
%  title = {{MS Windows NT} Kernel Description},
%  year = 1999,
%  url = {http://math2.uncc.edu/~ghetyei/courses/old/F07.3116/birkhofft.pdf},           
%}
\end{document}

