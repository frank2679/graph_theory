\documentclass[12pt]{article}
\usepackage{geometry}
\usepackage{amsmath}
\usepackage{amsthm}
\usepackage{graphicx}  % used to insert graph
\usepackage{subcaption}  % used to insert subgraph
\usepackage{amssymb}
\usepackage{enumitem}
\usepackage{fancyhdr}
\usepackage{tikz}
\usetikzlibrary{trees}

\geometry{left=2.5cm,right=3 cm, top=2.5cm,bottom=2.5cm}
\lhead{Chapter 3-1}
\chead{Graph Theory HW-5}
\rhead{Yang Hang, r03942126}

\begin{document}
\pagestyle{fancy}
\begin{enumerate}
\item[\textbf{3.1}] 
Because (1),(2),(3),(4) are equivalent, we only need to prove that (5),(6) are equivalent with one of $(1)\sim(4)$.
\begin{proof} 
$(1)\Rightarrow(5)$ : 
$G$ is a tree, i.e. $G$ is connected and has no cycle.  Because $G$ is connected, a path $x-y$ exists between $x$ and $y$. We can add an arbitrary edge $e: xy$. Then a cycle will be formed. 
%
$(5)\Rightarrow (1)$ : 
$G$ has no cycle. With an arbitrary edge $e: xy$, a cycle is formed. That means a path $x-y$ exits in $G$. $\Rightarrow$ $G$ is connected $\Rightarrow$ $G$ is a tree. 
\end{proof}
%
\begin{proof}
$(4)\Rightarrow (6)$ : Delete an arbitrary edge $e: xy$. Because only one path exists between two arbitrary vertexes $x$ and $y$. The path between $x$ and $y$ will be $xey$. So deleting the edge $e$, no path exists  between $x$ and $y$. $G$ will be unconnected. \\
$(6)\Rightarrow (4)$ : First prove $(6)\Rightarrow (1)$,then because (1) and (4) are equivalent, $(6)\Rightarrow (4)$ is proved.Following is proof of $(6)\Rightarrow (1)$.\\
Deleting an arbitrary edge $e: xy$, $G$ will be unconnected. $\Rightarrow$ $G$ has no cycle(\textbf{Property 3.4}). What's more, $G$ is connected, so $G$ is a tree. So$(6)\Rightarrow (1)$ is proved. 
\end{proof}
%
\item[\textbf{3.3}] 
Deleting an arbitrary vertex, $G$ will be a tree. $\Rightarrow$ $G$ has a cycle.(\textbf{Thm 3.7(5))}.\\ 
If $G$ has other vertexes except cycles, remove these vertexes, $G$ still have cycles, not a tree. That will be a contradiction.
Therefore, $G$ is just a cycle, containing n edges. 
% -----------------------------------------
\item[\textbf{3.5}] 
\begin{proof}
"$\Rightarrow$": $d_1, d_2,... d_n$ is a degree sequence of a tree. The tree has n-1 edges (textbf{Thm 3.7(2)}). 
Therefore, $\sum_{i=1}^{n} {d_i = 2(n-1)}$. \\
"$\Leftarrow$": $\sum_{i=1}^{n}{d_i}=2(n-1)$ $\Rightarrow$ $G$ has n-1 edges. Prove with induction on n. \\
$n = 2$, because $d_i \leq 1$, $d_1 = d_2 = 1$. $G$ with two vertexes and one edge, is of course a tree. \\
Assume $n = k$, there is a tree $T$ satisfying sequence $d$, then when $n = k+1$, $G$ has one more vertex and one more edge than tree $T$. Adding a vertex and an edge to tree $T$ will form a new tree $T^\prime$. Therefore, $G$ is $T^\prime$, a tree. 
\end{proof}
% -----------------------------------------
\item[\textbf{3.10}]
\begin{proof}
\begin{figure}[ht!]
%\begin{subfigure}{.5\textwidth}
  \centering
  \includegraphics[width=.5\linewidth]{hw5_1.png}
  \caption{Three components of Tree T}
  \label{fig1}
\end{figure} 
If there exist two non-adjacent medians, $x, y$. tree $T$ can be grouped into three groups, two groups rooting at $x$, $y$ respectively and a group between $x$ and $y$, $T_1,T_2,T_3$ as in Figure \ref{fig1}. $T_1, T_2, T_3$ have $n_1, n_2, n_3$ vertexes respectively, and $n_2 \geq 2$. In tree $T_2$, vertexes $a, b$ are directly connected with $x, y$ respectively. We have
%
\begin{align*}
S_T(x) &= S_{T_1}(x) + S_{T_2}(a) + S_{T_3}(y) + n_2 +n_3(2+d(a,b))\\
S_T(y) &= S_{T_1}(x) + S_{T_2}(b) + S_{T_3}(y) + n_2 +n_1(2+d(a,b))\\
S_T(a) &= S_{T_1}(x) + S_{T_2}(a) + S_{T_3}(y) + n_1 +n_3(1+d(a,b))\\
S_T(b) &= S_{T_1}(x) + S_{T_2}(b) + S_{T_3}(y) + n_3 +n_1(2+d(a,b))
\end{align*}
Then 
\begin{align*}
&(S_T(x)+S_T(y))-(S_T(a)+S_T(b)) = 2n_2 > 0 \\
&S_T(x)+S_T(y) > S_T(a)+S_T(b)
\end{align*}
which is a contradiction of definition of median. Therefore $x,y$ are adjacent.   \\
When $n_2 = 1$, in the same way, we can find that vertex $a$ in $T_2$, $S_T(a) < S_T(x) = S_T(y)$. So only one median exists. \\
Above all, only one median or two adjacent medians exit in tree $T$. 
\end{proof}
% -----------------------------------------
\item[\textbf{3.11}] 
\begin{proof}
Assume $\vert e(G) \vert=m$. To find a spanning tree $T_i$, $\vert e(T_i) \vert = n-1$, we need to remove m-n+1 edges of $G$. Two spanning trees are adjacent if they share $n-2$ edges. That means if two spanning trees are adjacent, they have only one different edge. \\
Let $\vartriangle(V_x,V_y)$ denote the number different edges between two spanning trees. When $\vartriangle(V_x,V_y) = 1$, two spanning trees $V_x,V_y$ are adjacent. Now, prove $d(V_x,V_y) = \vartriangle(V_x,V_y)$ by inducting on $\vartriangle$.\\ 
When $\vartriangle(V_x,V_y) = 1$, $d(V_x,V_y) = 1$. It is true.
Assume $\vartriangle(V_x,V_y) = k$ is right, $d(V_x,V_y) = k$. When $\vartriangle(V_x,V_y) = k+1$, according to induction, we can find a vertex $V_z$, such that $\vartriangle(V_x,V_z) = k, \vartriangle(V_y,V_z) = 1$, and $V_x,V_z$ are connected, $d(V_x,V_z) = k$. $V_z, V_y$ are directly connected. Then $d(V_x,V_y) = k+1$.
Therefore, $G^\prime$ is connected. \\ 
From above definition, a spanning tree is formed by removing $m-n+1$ edges from $G$. Therefore, the longest distance will be $m-n+1$, i.e. diameter will be m-n+1.  
\end{proof}

\end{enumerate}





\end{document}