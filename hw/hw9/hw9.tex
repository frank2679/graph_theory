\documentclass[12pt]{article}
\usepackage{geometry}
\usepackage{amsmath}
\usepackage{amsthm}
\usepackage{graphicx}  % used to insert graph
\usepackage{subcaption}  % used to insert subgraph
\usepackage{amssymb}
\usepackage{enumitem}
\usepackage{fancyhdr}
\usepackage{tikz}
\usetikzlibrary{trees}

%% for code listing
\usepackage{listings}
\usepackage{color}
\definecolor{dkgreen}{rgb}{0,0.6,0}
\definecolor{gray}{rgb}{0.5,0.5,0.5}
\definecolor{mauve}{rgb}{0.58,0,0.82}
\lstset{frame=tb,
  language=C,
  aboveskip=3mm,
  belowskip=2mm,
  showstringspaces=false,
  columns=flexible,
  basicstyle={\small\ttfamily},
  numbers=none,
  numberstyle=\tiny\color{gray},
  keywordstyle=\color{blue},
  commentstyle=\color{dkgreen},
  stringstyle=\color{mauve},
  breaklines=false, %true,
  breakatwhitespace=false, %true,
  tabsize=3
}


\geometry{left=2.5cm,right=3 cm, top=2.5cm,bottom=2.5cm}
\lhead{Chapter 4-1}
\chead{Graph Theory HW-9}
\rhead{Yang Hang, r03942126}

\begin{document}
\pagestyle{fancy}
\begin{enumerate}
% -----------------------------------------------------------------------------
\item[\textbf{5.2}] 
\item[(1)]
$|S\cap X| = a, |S\cap Y|=b, S, \overline{S} \neq \emptyset, a, b \in \mathbb{N} $. 
We have $0\leq a \leq m, 0 \leq b\leq n, 0<a+b<m+n $\\
$|[S,\overline{S}]| = a(n-b) + b(m-a) = an + bm -2ab$
\item[(2)] 
\begin{eqnarray}
\begin{aligned}
\kappa (K_{m,n}) &= \mathop{\min}{|[S,\overline{S}]|} \\
                 &= \mathop{\min}\limits_{0<a+b<m+n} (an+bm-2ab)\\
                 &= \mathop{\min} \lbrace m, n \rbrace
\end{aligned}
\end{eqnarray} 
Let $f(a,b) = an+bm -2ab$. We have
$\frac{\partial f}{\partial a} = n-2b,\frac{{\partial f}^2}{{\partial a}^2} = 0; 
\frac{\partial f}{\partial b} = m-2a,\frac{{\partial f}^2}{{\partial b}^2} = 0 $ \\
The domain of $f$ is a plain of $a,b$ except $(0,0),(m,n)$, as in Figure \ref{2}. From above, fixing $a$ or $b$, $f$ is monotone on the other variable. The minimum of $f$ will only get from the point in Figure \ref{2}.\\
Calculate value of these six points, ${(1,0),(0,1),(m-1,n),(m,n-1)}$ will let $f$ get the $m$ or $n$. 

\item[(3)]
\textit{Arbitrary seven edges will be a disconnecting set.}
$K3,3$ has 9 edges in total. After removing arbitrary 7 edges, only 2 edges remain. At most 3 vertices are connected in a component. Therefore, set of arbitrary 7 edges will be a disconnecting set. \\
\textit{No such an edge cut contains seven edges.}
Because each vertex has degree of 3, the size of any edge cut will be times of 3. 

\begin{figure}[ht!]
\begin{minipage}{.5\textwidth}
  \centering
  \includegraphics[width=.8\linewidth]{./figure/5_2.png}
  \caption{Domain of $f$}
  \label{2}
\end{minipage}%
\begin{minipage}{.5\textwidth}
  \centering
  \includegraphics[width=.5\linewidth]{./figure/5_4.png}
  \caption{n=5}
  \label{4}
\end{minipage}
\end{figure}
% -----------------------------------------------------------------------------
\item[\textbf{5.4}]
$\delta(G) \leq n(G) - 1 $.
When $\delta(G) = n(G)-1$, $G$ is $K_n$. It is true $\kappa(K_n) = \delta(K_n)$.\\
According to \textbf{Thm 5.4}, we have $\kappa(G) \leq \delta(G)$. When $\delta(G) = n(G) - 2$, two disconnected vertices $x,y$ always exists. Let $S = V(G)\backslash {x,y}$. $S$ will be a vertex cut. So $\kappa (G)\geq n(G)-2 = \delta(G)$. Then $\kappa(G) = \delta(G)$.\\
With the result above, $\kappa(G) = \delta(G)$ given $\delta(G)\geq n(G) -2$.\\
When n=5, graph of Figure \ref{4} satisfies $\delta(G) = n-3, \kappa(G) < n-3$.

% ----------------------------------------------------------------------------------------------
\item[\textbf{5.6}]
\begin{proof}
$\Delta(G) \leq 3$ means $deg(v) \leq 3, v\in G$. Similar with proof of \textbf{Thm 5.6}, let $S$ be the smallest vertex cut. We just need to find a disconnecting set $F$ with the same size of $S$. Then we will finish proof because $\kappa(G) \leq \kappa^\prime(G)$.\\
First, if $\exists v: deg(v)=1$, $\kappa(G) = \kappa^\prime(G) = 1$. Otherwise, let $H_1, H_2$ be two components of $G-S$. For $v\in S$, because $S$ is the smallest vertex cut. $v$ has neighbors in both $H_1, H_2$. In the proof of \textbf{Thm 5.6}, for $v: deg(v) = 3$, we denote the set as $S_1$. We have selected $|S_1|$ edges as edges of $F$. Then We discuss about $v: deg(v) = 2$ denoted by set $S_2$. For $v: deg(v)=2$, $v$ has one neighbor in $H_1, H_2$ respectively. Then we select one edge of the two as edge in $F$. Therefore, $F$ of size $|S_1+S_2|=|S|$ will cut all paths between $H_1$ and $H_2$. 
\end{proof}
% ----------------------------------------------------------------------------------------------
\item[\textbf{5.8}]
We know that number of edges of a cycle with $n$ vertices is $n$. Block of edge contains two vertices and one edge. Combination of blocks of edge will form a tree, which has $n-1$ edges. Combinations of blocks of cycle will have $n+k$ edges, $k$ is the number of blocks. The cactus with the most edges must contains most blocks of cycle. So cycle of 3 nodes will be the choice. \\
As in Figure \ref{8}, all components share a common vertex. The number of edges will be $\lfloor 3(n-1)/2 \rfloor$.


\begin{figure}[ht!]
\begin{minipage}{.5\textwidth}
  \centering
  \includegraphics[width=.5\linewidth]{./figure/5_8.png}
  \caption{Cactus with most edges}
	\label{8}
\end{minipage}%
\begin{minipage}{.5\textwidth}
  \centering
  \includegraphics[width=.8\linewidth]{./figure/5_10.png}
  \caption{Case of $P_i\cap Q_j \neq \emptyset$}
	\label{10}
\end{minipage}
\end{figure}
% ----------------------------------------------------------------------------------------------
\item[\textbf{5.10}]
\textit{Sufficiency}. Sufficiency is obvious. Remove arbitrary vertex $x$, we can always find a path between an arbitrary pair of vertex, i.e. connected. With \textbf{Thm 5.10(1)}, $G$ is 2-connected.\\
\textit{Necessity}.
With \textbf{Thm 5.10(2)}, for any $x,y$, we can find two disjoint paths $x-y$, denoted by $P_1,P_2$. Similarly, two disjoint paths $y-z$  are denoted by $Q_1,Q_2$. If $P_i\cap Q_j = \emptyset, i,j = 1,2$, such a path $x-z$ that go through $y$ exists as $P_i+Q_j$.\\
If $P_i\cap Q_j \neq \emptyset, \forall i,j = 1,2$, Figure \ref{10} as an example, we can also find a path $x-z$ going through $y$. For example, path $x$ go along $P_2$ to $y$, then along $P_1$ to $b$, finally along $Q_1$ to $z$ is such a path.

\end{enumerate}

\end{document}

