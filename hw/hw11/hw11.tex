\documentclass[12pt]{article}
\usepackage{geometry}
\usepackage{amsmath}
\usepackage{amsthm}
\usepackage{graphicx}  % used to insert graph
\usepackage{subcaption}  % used to insert subgraph
\usepackage{amssymb}
\usepackage{enumitem}
\usepackage{fancyhdr}
\usepackage{tikz}
\usetikzlibrary{trees}

%% for code listing
\usepackage{listings}
\usepackage{color}
\definecolor{dkgreen}{rgb}{0,0.6,0}
\definecolor{gray}{rgb}{0.5,0.5,0.5}
\definecolor{mauve}{rgb}{0.58,0,0.82}
\lstset{frame=tb,
  language=C,
  aboveskip=3mm,
  belowskip=2mm,
  showstringspaces=false,
  columns=flexible,
  basicstyle={\small\ttfamily},
  numbers=none,
  numberstyle=\tiny\color{gray},
  keywordstyle=\color{blue},
  commentstyle=\color{dkgreen},
  stringstyle=\color{mauve},
  breaklines=false, %true,
  breakatwhitespace=false, %true,
  tabsize=3
}


\geometry{left=2.5cm,right=3 cm, top=2.5cm,bottom=2.5cm}
\lhead{Chapter 6-1}
\chead{Graph Theory HW-11}
\rhead{Yang Hang, r03942126}

\begin{document}
\pagestyle{fancy}
\begin{enumerate}
% -----------------------------------------------------------------------------
\item[\textbf{6.2}] 
A torus can be unfolded into a plane that vertex in bottom and top is the same while that in left edge and right edge is the same. Figure \ref{2} is an example.
\begin{figure}[htbp]
	\centering
	\includegraphics[width=.3\linewidth]{./figure/6_2.png}
	\caption{K7 embedded into a torus \\ ref[1]}
	\label{2}
\end{figure} 

% -----------------------------------------------------------------------------
\item[\textbf{6.3}]
\begin{proof}
A plane can be written in a form of $ax+by+cz+d=0$. Substitute $(x,y,z)=(t,t^2,t^3)$ into the equation obtaining a cubic equation. 
The cubic equation has 3 different solutions at most. Therefore, no such four vertices exist that they all lay on the same plane.
\end{proof}

% ----------------------------------------------------------------------------------------------
\item[\textbf{6.4}]
\item[(1)]
\begin{proof}
\textit{A polygon with $ n \leq 5$, there is a vertex inside $G$ can see all vertices inside the polygon.}
It is trivial when $n=3$. When $n=4$, vertex on the shorter diagonal line works. When $n=5$, sum of interior angle is $540^\circ$ degree.
If no angle extends $180^{\circ}$, it is convex. Such vertex exists.
If one angle extends $180^{\circ}$, vertices inside the triangle of the obtuse angle and other two disjoint vertices are such vertices.
If two angle extends $180^{\circ}$, vertices in the shadow of the Figure \ref{4}.
\end{proof}
\begin{figure}[htbp]
	\centering
	\includegraphics[width=.3\linewidth]{./figure/6_4.png}
	\caption{Polygon(n=5) with two obtuse angles}
	\label{4}
\end{figure}
\item[(2)]
\begin{proof}
\textit{Any planar graph has a straight-line drawing.}
Prove by induction on the number of vertices. 
Without loss of generality assume that $G$ is maximally planar(means adding any edge will make $G$ not planar). It is because if maximally planar graph is can be straight-line drawn, removing any edges the graph can be still straight-line drawn.\\
According to \textbf{Corollary 6.8}, planar graph $G$ always has vertex $\delta(v)\leq 5$. Remove $v$ and triangulate the face that is created by it's removal to create $G^\prime$. $G^\prime$ is a maximally planar graph which has fewer vertices than $G$. With induction hypothesis, $G^\prime$ can be straight-line drawn.
Then to turn $G^\prime$ into a straight line drawing of $G$ first erase allthe edges which were used to triangulate the face the removal $v$ created. Next by \textbf{(1)}, we know there is a vertex "see" all vertices in the face, since there are 5 vertices on the face. We add $v$ at such a location, and add the lines between it and the vertices of the face. This gives a straight line drawing of $G$. 

\end{proof}
% ----------------------------------------------------------------------------------------------
\item[\textbf{6.6}]
\begin{proof}
If a graph is isomorphic with its dual, it is connected, because any graph's dual graph is connected. 
With Euler polyhedral formula, we have $n-e+f=2$. Since isomorphic, $n=f$. Then we have $e=2n-2$.
\end{proof}
Figure \ref{6} is an example of $n=5$ that the original plane graph is isomorphic with its dual. Its $n-1$ nodes form a cycle and the left one is in the center joining all other nodes. 
\begin{figure}[htbp]
	\centering
	\includegraphics[width=.3\linewidth]{./figure/6_6.png}
	\caption{$n=5$, Plane graph isomorphic with its dual}
	\label{6}
\end{figure} 

% ----------------------------------------------------------------------------------------------
\item[\textbf{6.7}]
\begin{proof}
According to \textbf{Corollary 6.8}, for any planar graph $G$, $\delta(G) \leq 5$. Therefore, There is no 6-connected planar graph. 
\end{proof}

\item[\textbf{6.9}]
\begin{proof}
\textit{A planar graph with n nodes and girth of k has at most $(n-2)\frac{k}{k-2}$.} Each edge is calculated by both sides of faces, while each face has at least k edges. We have $2e \geq kf$. Substitute $n-e+f=2$ into the inequality. We will have $e\geq (n-2)\frac{k}{k-2}$.

\textit{Petersen graph is not planar graph.}
Petersen graph $G$, $n=10, e=15, k=5$. According to above, $e \leq 13$, which results in contradiction. Petersen graph is not planar graph.
\end{proof}

\end{enumerate}
\textbf{Reference} 

[1] http://www3.math.tu-berlin.de/geometrie/Lehre/WS12/MathVis/resources/projects\\/loeweSiegSlides.pdf


\end{document}

