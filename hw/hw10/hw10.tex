\documentclass[12pt]{article}
\usepackage{geometry}
\usepackage{amsmath}
\usepackage{amsthm}
\usepackage{graphicx}  % used to insert graph
\usepackage{subcaption}  % used to insert subgraph
\usepackage{amssymb}
\usepackage{enumitem}
\usepackage{fancyhdr}
\usepackage{tikz}
\usetikzlibrary{trees}

%% for code listing
\usepackage{listings}
\usepackage{color}
\definecolor{dkgreen}{rgb}{0,0.6,0}
\definecolor{gray}{rgb}{0.5,0.5,0.5}
\definecolor{mauve}{rgb}{0.58,0,0.82}
\lstset{frame=tb,
  language=C,
  aboveskip=3mm,
  belowskip=2mm,
  showstringspaces=false,
  columns=flexible,
  basicstyle={\small\ttfamily},
  numbers=none,
  numberstyle=\tiny\color{gray},
  keywordstyle=\color{blue},
  commentstyle=\color{dkgreen},
  stringstyle=\color{mauve},
  breaklines=false, %true,
  breakatwhitespace=false, %true,
  tabsize=3
}


\geometry{left=2.5cm,right=3 cm, top=2.5cm,bottom=2.5cm}
\lhead{Chapter 5-2}
\chead{Graph Theory HW-10}
\rhead{Yang Hang, r03942126}

\begin{document}
\pagestyle{fancy}
\begin{enumerate}
% -----------------------------------------------------------------------------
\item[\textbf{5.11}] 
Do degree ear decomposition; then no link exists between neighbor of joint vertex on cycle or previous ear and the neighbor on current ear,i.e. no edge $ac, cd, eh, gh$ as in Figure \ref{11}. 
Otherwise, a large cycle $a-bca$ exists.\\
What's more, since $G$ is claw-free, $ad, eg$ must exist. Then we can find disjoint graph $P_3$ as follow. \\
Select graph $P_3$ from ear $P_n$ to $P_0$. $k = 0,1,2,...,$ For ear $P_i$, if $n(P_i) = 3(k+1)$, we can select $k$ disjoint graph $P_3$. Then the neighbor of joint vertex will help hold 2-connectivity for remaining graph.\\
If $n(P_i) = 3k+1$, select $k$ graph $P_3$ from one endpoint. The other endpoint remains. \\
If $n(P_i) = 3k + 2$, select $k$ graph $P_3$ among the middle vertices of ear $P_i$. \\
Use the method above, 2-connectivity holds after selecting one ear. No vertex is "wasted". Therefore, we can find $\lfloor n(G)/3\rfloor $ graph $P_3$. 
\begin{figure}[hbp!]
	\centering
	\includegraphics[width=.3\linewidth]{./figure/5_11.png}
	\caption{Greedy ear decomposition}
	\label{11}
\end{figure} 

% -----------------------------------------------------------------------------
\item[\textbf{5.13}]
\begin{proof}
We prove the contrary that \textit{if each block of $G$ is not an edge or odd cycle, even cycle exits in $G$}. Proof are in two parts.\\
If $G$ is 1-connected, such an edge exists that it is the only edge connecting two blocks. This edge is the block of edge. It contradicts the condition.\\
$G$ must be 2-connected; then it has an ear decomposition. For ear $P_0,P_1$, two uncover cycles are odd cycles. The big cycle will be an even cycle.\\ 
\end{proof}

% ----------------------------------------------------------------------------------------------
\item[\textbf{5.14}]
\begin{proof}
We prove by find $k$ disjoint paths between arbitrary $x,y$.\\
$k=1$ is trivial. For $k\leq 2$, we can draw the $Q_k$ as $k-1$ slices. The slice is a cycle of 4 vertices, and each slice is connected with all other slices by links between corresponding vertices, $Q_4$ in Figure \ref{14} as an example.\\
For $x,y$ are in the same slice. We can find two disjoint paths among the slice. And find other $k-2$ disjoint paths by going from $x$ to other $k-2$ slices then going through the slice to the counterpart vertex of $y$ at last reaching $y$. Total in $k$ paths. \\
For $x,y$ not in the same slice, we can find three disjoint paths between the two slices of $x,y$ reaching $y$. And find another $k-3$ disjoint paths by going from $x$ to other slices first then do as former situation. \\
Above all, $Q_k$ is $k$-connected. 
\begin{figure}[htbp!]
	\centering
	\includegraphics[width=.3\linewidth]{./figure/5_14.png}
	\caption{$Q_4$}
	\label{14}
\end{figure} 
\end{proof}
% ----------------------------------------------------------------------------------------------
\item[\textbf{5.17}]
\item[(1)]
\begin{proof}
Since $G$ is two-connected, $\delta(G) \leq 2$. We prove when $\delta(G)\leq 3$, $G$ is not minimally 2-connected.\\
Do ear decomposition. In last ear $P_n$, for $v \in P_n\backslash {endpoints}$, since $\delta(G)\leq 3$, an $v$-induced edge $e$ exists that $e\not\in P_i$. So $G-e$ is still 2-connected, which contradicts definition of minimal 2-connected. 
\end{proof}
\item[(2)]
\begin{proof}

\end{proof}

% ----------------------------------------------------------------------------------------------
\item[\textbf{5.10}]
\textit{Sufficiency}. Sufficiency is obvious. Remove arbitrary vertex $x$, we can always find a path between an arbitrary pair of vertex, i.e. connected. With \textbf{Thm 5.10(1)}, $G$ is 2-connected.\\
\textit{Necessity}.
With \textbf{Thm 5.10(2)}, for any $x,y$, we can find two disjoint paths $x-y$, denoted by $P_1,P_2$. Similarly, two disjoint paths $y-z$  are denoted by $Q_1,Q_2$. If $P_i\cap Q_j = \emptyset, i,j = 1,2$, such a path $x-z$ that go through $y$ exists as $P_i+Q_j$.\\
If $P_i\cap Q_j \neq \emptyset, \forall i,j = 1,2$, Figure \ref{10} as an example, we can also find a path $x-z$ going through $y$. For example, path $x$ go along $P_2$ to $y$, then along $P_1$ to $b$, finally along $Q_1$ to $z$ is such a path.

\end{enumerate}

\end{document}

