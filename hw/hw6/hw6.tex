\documentclass[12pt]{article}
\usepackage{geometry}
\usepackage{amsmath}
\usepackage{amsthm}
\usepackage{graphicx}  % used to insert graph
\usepackage{subcaption}  % used to insert subgraph
\usepackage{amssymb}
\usepackage{enumitem}
\usepackage{fancyhdr}
\usepackage{tikz}
\usetikzlibrary{trees}

%% for code listing
\usepackage{listings}
\usepackage{color}
\definecolor{dkgreen}{rgb}{0,0.6,0}
\definecolor{gray}{rgb}{0.5,0.5,0.5}
\definecolor{mauve}{rgb}{0.58,0,0.82}
\lstset{frame=tb,
  language=C,
  aboveskip=3mm,
  belowskip=2mm,
  showstringspaces=false,
  columns=flexible,
  basicstyle={\small\ttfamily},
  numbers=none,
  numberstyle=\tiny\color{gray},
  keywordstyle=\color{blue},
  commentstyle=\color{dkgreen},
  stringstyle=\color{mauve},
  breaklines=false, %true,
  breakatwhitespace=false, %true,
  tabsize=3
}


\geometry{left=2.5cm,right=3 cm, top=2.5cm,bottom=2.5cm}
\lhead{Chapter 3-2}
\chead{Graph Theory HW-6}
\rhead{Yang Hang, r03942126}

\begin{document}
\pagestyle{fancy}

\begin{enumerate}
\item[\textbf{3.13}] 
\begin{enumerate}
\item Use BFS to find the distance between $x,y$. Following is the pseudocode.   
%% code listing
\begin{lstlisting}
BFS(G,x,y){
	initialize Q;
	// x as start; gray means to be visited
	x.height = 0; x.status = gray; 
	// white means not visited; black means visited
	other vertexes are all white;  
	Q.push(x);
	while(Q is not empty)
		u = Q.pop();
		for (v in neighbors of u)
			if (v.status = white)
				v.status = gray;
				v.height = u.height + 1; 
				Q.push(v);
				if(v == y)
					return v.height; // v's height is distance(x,y). 
		u.status = blace;
}
\end{lstlisting}
\item The necessary and sufficient condition of that a graph is a bipartite graph is the length of the cycles in graph must be even.Use DFS to find the length of a cycle.
\begin{lstlisting}
DFS(G,s){
	initialize Q; // Q is a stack
	Q.push(s); s.status = gray; // s is the start
	all other nodes are white;
	while(s is not empty)
		u = Q.pop();
		for(v in neighbors of u)
			if (v.status == white)			
				v.status = gray;
				u.height = v.height + 1;			
				Q.push(u);
			else // i.e. a cycle exists
				len_cycle = u.height - v.height + 1; // len_cycle = d(u,v) + 1
		if(len_cycle == odd)
			return G is not bipartite;
		u.status = black;
	return G is bipartite; 
}
\end{lstlisting}
A cycle is containing a path from $u$ to $v$ and the edge $uv$. The distance between $u,v$ is $u.height - v.height$. So $LengthofCycle = u.height - v.height + 1$.
\end{enumerate}
% -----------------------------------------
\item[\textbf{3.14}] 
We can separate spanning trees into 5 categories according to the number of deleting edges among the middle 4 edges as in Figure \ref{14}. Then calculate number of spanning trees respectively. \\
\begin{figure}[hbp!]
	\centering
	\includegraphics[width=.8\linewidth]{./figure/3_14.png}
	\caption{Five categories}
	\label{14}
\end{figure} 
In category a, to form a tree, we need to remove one edge in the three small cycle. What's more, the remaining 7 edges form a large cycle. We also need to remove one edge among the 4 edges because the other three middle edges can not be removed in category a. We can choose one from the four middle edges. So
$\tau(a) = \binom {4}{1}^5 $
Similarly, $\tau(b) = \binom {4}{1}^3\binom {8}{1} $,
$\tau(c) = 2\binom {4}{1}^2\binom {8}{1} $,
$\tau(d) = 4 \times 12\times 4 $,
$\tau(e) = 16 $.
Therefore, $\tau(G) = \tau(a)+\tau(b)+\tau(c)+\tau(d)+\tau(e)=2000$.
% -----------------------------------------
\item[\textbf{3.16}] 
\begin{proof}
\begin{figure}[hbtp!]
\begin{subfigure}{.5\textwidth}
  \centering
  \includegraphics[width=.8\linewidth]{./figure/3_16-1.png}
  \caption{$G_n$}
  \label{16-1}
\end{subfigure}%
\begin{subfigure}{.5\textwidth}
  \centering
  \includegraphics[width=.8\linewidth]{./figure/3_16-2.png}
  \caption{$(G-e_i), i = 1,2,3$}
  \label{16-2}
\end{subfigure}
\begin{subfigure}{.5\textwidth}
  \centering
  \includegraphics[width=.8\linewidth]{./figure/3_16-3.png}
  \caption{$(G-e_4)$ and its equivalent graph $G^\prime$}
  \label{16-3}
\end{subfigure}
\begin{subfigure}{.5\textwidth}
  \centering
  \includegraphics[width=.8\linewidth]{./figure/3_16-4.png}
  \caption{Derive of Contraction of $(G-e_4)$, $G_{n-1}, G^{\prime\prime}$}
  \label{16-4}
\end{subfigure}
\caption{$G_n$ and its subgraphs}
\label{16}
\end{figure}
When $n = 1$, $\tau(G_1) = 1$.
When $n = 2$, $\tau(G_2) = 4$.\\
When $n \geqslant 3$, in Figure \ref{16-1}, $\tau(G_n) = \tau(G_n-e_1) +\tau(G_n-e_2) + \tau(G_n-e_3) +\tau(G_n-e_4) $. According to Figure \ref{16-2}, it is obvious that $\tau(G_n-e_1) = \tau(G_n-e_2) = \tau(G_n-e_3) = \tau(G_{n-1})$. \\
From Figure \ref{16-3}, $\tau(G_n-e_4) = \tau(G^\prime)$. And with contract, from Figure \ref{16-4}, we can derive that $\tau(G^\prime) = \tau(G_{n-1}) - \tau(G^{\prime\prime})$.
And it is obvious that $\tau(G^{\prime\prime}) = \tau(G_{n-2}) $.
So $\tau(G_n-e_4) = \tau(G_{n-1}) - \tau(G_{n-2}) $.\\
Therefore, $\tau(G_n) =  4\tau(G_{n-1}) - \tau(G_{n-2})$.
\end{proof}
% -----------------------------------------
\item[\textbf{3.19}]
\textbf{Lemma}: An edge connecting two trees forms another tree. \\
The lemma is obviously true because the formed graph is connected and has no cycles. 
\begin{proof}
$T,T^\prime$ are two spanning trees of graph $G$.
We can group the vertexes of graph $G$ as Figure \ref{19-1}, $T_i, T_j$ are two trees rooted at $v_i, v_j$ respectively. In Figure \ref{19-2}, vertexes of groups is the same with $T$, but not rooted at $v_i,v_j$. \\
Then, $\forall e:v_iv_j \in E(T) \backslash E(T^\prime)$, there exists a path $P: v_i-v_j$ in $T^\prime$. 
And $\exists e^\prime = e_k$ such that $v_k \in T_i^\prime, v_{k+1} \in T_j^\prime$, i.e. $v_k \in T_i, v_{k+1} \in T_j$. 
We are easy to prove $e_k \not\in T$. Because if $e_k \in T$, a cycle will be formed in $T$.\\
Therefore, according to lemma, $T-e+e^\prime, T^\prime+e-e^\prime $ are also spanning trees. 
\begin{figure}[hb!] % h: current place, t: title of page, b: bottom of page, p: floting page
\begin{subfigure}{.5\textwidth}
  \centering
  \includegraphics[width=.8\linewidth]{./figure/3_19-1.png}
  \caption{Spanning Tree T}
  \label{19-1}
\end{subfigure}%
\begin{subfigure}{.5\textwidth}
  \centering
  \includegraphics[width=.8\linewidth]{./figure/3_19-2.png}
  \caption{Spanning Tree $T^\prime$}
  \label{19-2}
\end{subfigure}
\caption{Spanning Trees of $G$}
\label{19}
\end{figure}

\end{proof}
% -----------------------------------------
\item[\textbf{3.21}] 
\begin{proof}
Proof contains two parts. 
\begin{enumerate}
\item Prove $T$ is a spanning tree \\
According to the algorithm, a node is added to S one by one. At start, S contains one node. It is a tree, named $T$. And with a node added to $T$, because the node is also a tree, $T$ adding the node will also be a tree by a edge connecting the node with $T$. 
This procedure continues until $S = V(G)$. $T$ is always a tree.
\item Prove $T$ is a minimum spanning tree\\
\begin{figure}[htp!]
	\centering
	\includegraphics[width=.35\linewidth]{./figure/3_21.png}
	\caption{Spanning Tree}
	\label{21}
\end{figure} 
Let $T^*$ is a minimum spanning tree(MST) which shares most common edges with $T$. If $T=T^*$, $T$ will be a MST.

If not, $\exists e \in E(T)\backslash E(T^*), e = uv$. Then a path $P$ exists between $u,v$. And $\exists e^\prime \in P$ as in Figure \ref{21} such that one endpoint is in $S$, the other in $\overline{S} $. What's more, $e^\prime \in E(T^*)\backslash E(T)$, (if not, cycle will exist in $T$). Then according to \textbf{Property 3.23}, $T^\prime = T^* - e^\prime + e $ is another spanning tree. And since Prim's algorithm always find the least weighted edge between $S$ and $\overline{S}$, $w(e) \leq w(e^\prime)$. Then $w(T^\prime) \leq w(T^*)$. It means $T^\prime$ is also a MST. But $T^\prime$ shares one more common edge with $T$, which is a contradiction to assumption of $T^*$. \\
Therefore, $T$ is a MST.  
\end{enumerate}
\end{proof}

\end{enumerate}





\end{document}
