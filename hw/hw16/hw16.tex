\documentclass[12pt]{article}
\usepackage{geometry}
\usepackage{amsmath}
\usepackage{amsthm}
\usepackage{graphicx}  % used to insert graph
\usepackage{subcaption}  % used to insert subgraph
\usepackage{amssymb}
\usepackage{enumitem}
\usepackage{fancyhdr}
\usepackage{tikz}
\usetikzlibrary{trees}

%% for code listing
\usepackage{listings}
\usepackage{color}
\definecolor{dkgreen}{rgb}{0,0.6,0}
\definecolor{gray}{rgb}{0.5,0.5,0.5}
\definecolor{mauve}{rgb}{0.58,0,0.82}
\lstset{frame=tb,
  language=C,
  aboveskip=3mm,
  belowskip=2mm,
  showstringspaces=false,
  columns=flexible,
  basicstyle={\small\ttfamily},
  numbers=none,
  numberstyle=\tiny\color{gray},
  keywordstyle=\color{blue},
  commentstyle=\color{dkgreen},
  stringstyle=\color{mauve},
  breaklines=false, %true,
  breakatwhitespace=false, %true,
  tabsize=3
}


\geometry{left=2.5cm,right=3 cm, top=2.5cm,bottom=2.5cm}
\lhead{Chapter 8-2}
\chead{Graph Theory HW-16}
\rhead{Yang Hang, r03942126}

\begin{document}
\pagestyle{fancy}
\begin{enumerate}
% -----------------------------------------------------------------------------
\item[\textbf{8.7}]
\begin{proof}
$(1)\Rightarrow (2)$: 
Because $d_k \geq n/2$, $deg(u)+deg(v) \geq n$.
\end{proof}

\begin{proof}
$(2)\Rightarrow (3)$: Prove the contrapositive. If $\exists k<n/2, d_k\leq k$. We claim a contradiction that $\exists uv\not \in E(G), deg(u)+deg(v)<n$. 
Assume $v_k$ as the smallest $k$ for $d_k\leq k$. Then $k-1 < d_{k-1}\leq k$, i.e., $d_{k-1}= d_k=k$. We prove that such $uv$ mentioned above exists under worst case. For $1\leq i\leq k$, it must be a clique, if not, such an edge $uv$ exists among them. 

\setlength\parindent{24pt} The remaining $n-k$ vertices form a clique with degree of $n-k+1$. From above, for $v_i, 1\leq i \leq k$, there are at most $k$ edges join them with $v_j, j > k$. Because edges joining $v_i, 1\leq i\leq k$ with $v_j, k<j\leq n$ join in backward order(from $v_n$ to $v_{k+1}$) and $k<n/2$, no edge will join $v_{k+1}$with $v_i, 1\leq i\leq k$, i.e.,$deg(v_{k+1})=n-k-1$. Then $v_kv_{k+1}, deg(v_k)+deg(v_{k+1})=n-1<n$, which forms the contradiction. 
\end{proof}

\begin{proof}
$(3)\Rightarrow (4)$: 
From condition, we have $i\geq\lfloor n/2 \rfloor, d_i\geq n/2$. For $d_j\leq j, d_k< k$, $j\geq n/2, k\geq n/2$. Then $d_j+d_k\geq n$.
\end{proof}

\begin{proof}
$(4)\Rightarrow (5)$:
We prove the contrapositive. If $d_k\leq k<n/2$ and $d_{n-k}< n-k$, since $n-k > k$, then $d_k+d_{n-k} < n$, which is a contradiction.
\end{proof}

\begin{proof}
$(5)\Rightarrow (6)$: 
We prove the contrapositive. If $i,j$ satisfy those conditions, we assume $d_i+d_j<n$. Then contradiction against (5) should be found. Without loss of generality, assume $i<j$. 

\setlength\parindent{24pt} If $i+j=n$, $i<n/2$ which satisfies condition of (5). Then $d_{n-i}=d_j\geq j$, a contradiction. If $i+j>n$, we assume $d_i=i$. Since $2d_i\leq d_i+d_j< n$, $i<n/2$. Then $n=i+(n-i)\leq d_i+d_{n-i}\leq d_i+d_j< n$, a contradiction. For $d_i<i$, since $d_i$ is increasing sequence, the property also hold.
\end{proof}

\begin{proof}
$(6)\Rightarrow (7)$: 
Assume $i,j$ as $\max\lbrace i+j\rbrace$ which satisfies $1\leq i, j \leq n, v_iv_j\not \in E(G)$. Without lack of generality, we assume $i<j$. 
$v_j \ldots v_n$ form a clique, otherwise a bigger pair exists. And $d_j\geq n-i-1,d_i\geq n-j$. 
Then, if $d_i>i$, $d_i+d_j>n-1$, i.e., $d_i+d_j\geq n$. If $d_j\geq j$, $d_i+d_j\geq n$. If $i+j<n$, $d_i+d_j\geq 2n-(i+j)-1\geq n$. Hence, $v_i, v_j$ will be linked according to definition of Hamiltonian closure. 

\setlength\parindent{24pt} After adding the edge $v_iv_j$ forming $G^\prime$, $G^\prime$ still satisfy the \textbf{Las Vergnas Condition}. Iteratively adding such edges yields $K_n$.
\end{proof}
% -----------------------------------------------------------------------------
\item[\textbf{8.9}] 
\begin{proof}
\textit{Every unique 3-edge-coloring 3-regular graph has Hamiltonian cycle.}
Graph $G$ is uniquely 3-edge-colored and 3-regular. One color class will be a perfect matching. Two classes named $H$ is 2-factor, which is union of cycles. 
If $H$ is just a single cycle, it will be a Hamiltonian cycle. If not, choose one of the cycle, switching the color yields another 3-edge-coloring, a contradiction against the uniquely 3-edge-coloring. 
\end{proof}

% -----------------------------------------------------------------------------
\item[\textbf{8.10}]
\begin{proof}
\textit{A plane triangulation has a vertex partition into two sets inducing forests if and only if the dual is Hamiltonian.} Let the plane triangulation be $G$, $G^*=F$.

\setlength\parindent{24pt} "$\Leftarrow$" $F$ has a Hamiltonian cycle $C$. $H$ consisting of $C$ and some edges of $F$ inside $C$ is outerplanar. Then the $H^*$ without the vertex of outer face of $H$ is one set of $G$, inducing forests. This argument also applies to the graph consisting of cycle $C$ and edges of $F$ outside $C$. 

\setlength\parindent{24pt} "$\Rightarrow$" We can always find a vertex partition to get two trees $S,\bar{S}$. Then $[S,\bar{S}]$ is a bond. Dual of edges $[S,\bar{S}]$ will form a cycle. We claim that this cycle is a spanning cycle. 
Since triangulation $G$ has $3n-6$ edges and two trees consist of $n-2$ edges, $\vert [S,\bar{S}]\vert=2n-4$ and $n(F)=f(G)=2n-4$. Such a cycle is Hamiltonian cycle. 
\end{proof}


\begin{figure}[ht!]
  \centering
  \includegraphics[width=.7\linewidth]{./figure/8_11.png}
 % \caption{Hamiltonian cycle in 8x8}
  \label{8_2}
\end{figure}
%----------------------------------------------------------------------------------------------
\item[\textbf{8.11}]
\item[(a)]

\begin{proof} 
Both the graphs are non-Hamiltonian graphs. The \textbf{Grinberg Theorem} requires that $\sum_i(i-2)(f^\prime_i-f^{\prime\prime}_i)=0$. The graph $G_1$ has six 4-edge faces and one 8-edge face. Then we have $2(f^\prime_4-f^{\prime\prime}_4)+6(f^\prime_8-f^{\prime\prime}_8)=0$. Rewrite the equation to get $(f^\prime_4-f^{\prime\prime}_4)+3(f^\prime_8-f^{\prime\prime}_8)=0$. And $f^\prime_4+f^{\prime\prime}_4=6, f^\prime_8+f^{\prime\prime}_8=1$. The first term will be even while the second term is odd, so the equation will not be satisfied.

\setlength\parindent{24pt} For the second graph $G_2$, redraw it as a plan graph.
$G_2$ has three 3-edge faces and six 6-edge faces. The equation $(f^\prime_3-f^{\prime\prime}_3)+4(f^\prime_6-f^{\prime\prime}_6)=0$ will not hold. Because $f^\prime_3+f^{\prime\prime}_3=3, f^\prime_6+f^{\prime\prime}_6=6$, $f^\prime_3-f^{\prime\prime}_3=\pm1,\pm3$ and $f^\prime_6-f^{\prime\prime}_6=0,\pm2, \pm4$. The first term of the equation is odd while the second is even. 
\end{proof}

% ----------------------------------------------------------------------------------------------
\item[\textbf{8.13}]

\begin{proof}
Similar with problem \textbf{8.11}, Hamiltonian graph requires the equation $3(f^\prime_5-f^{\prime\prime}_5)+6(f^\prime_8-f^{\prime\prime}_8)+7(f^\prime_9-f^{\prime\prime}_9)=0$. $f^\prime_5+f^{\prime\prime}_5=21, f^\prime_8+f^{\prime\prime}_8=3, f^\prime_9-f^{\prime\prime}_9=1$. Let $x=f^\prime_5-f^{\prime\prime}_5, y=f^\prime_8-f^{\prime\prime}_8, z=f^\prime_9-f^{\prime\prime}_9$. $x$ is odd, $y$ is odd, $z=\pm1$. Rewrite the equation, $3(x+2y)+7z=0$. The first term is multiple of 3, while the second is $\pm7$. The equation doesn't hold.
\end{proof}
\end{enumerate}


\end{document}

