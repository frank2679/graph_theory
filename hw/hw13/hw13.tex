\documentclass[12pt]{article}
\usepackage{geometry}
\usepackage{amsmath}
\usepackage{amsthm}
\usepackage{graphicx}  % used to insert graph
\usepackage{subcaption}  % used to insert subgraph
\usepackage{amssymb}
\usepackage{enumitem}
\usepackage{fancyhdr}
\usepackage{tikz}
\usetikzlibrary{trees}

%% for code listing
\usepackage{listings}
\usepackage{color}
\definecolor{dkgreen}{rgb}{0,0.6,0}
\definecolor{gray}{rgb}{0.5,0.5,0.5}
\definecolor{mauve}{rgb}{0.58,0,0.82}
\lstset{frame=tb,
  language=C,
  aboveskip=3mm,
  belowskip=2mm,
  showstringspaces=false,
  columns=flexible,
  basicstyle={\small\ttfamily},
  numbers=none,
  numberstyle=\tiny\color{gray},
  keywordstyle=\color{blue},
  commentstyle=\color{dkgreen},
  stringstyle=\color{mauve},
  breaklines=false, %true,
  breakatwhitespace=false, %true,
  tabsize=3
}


\geometry{left=2.5cm,right=3 cm, top=2.5cm,bottom=2.5cm}
\lhead{Chapter 7-1}
\chead{Graph Theory HW-13}
\rhead{Yang Hang, r03942126}

\begin{document}
\pagestyle{fancy}
\begin{enumerate}
% -----------------------------------------------------------------------------
\item[\textbf{7.5}] 
\textit{$G$ is m-colorable if and only if $\alpha (G\square K_m) \geq n(G)$.}
\begin{proof} 
\textit{Necessity} 
$\chi(G\square K_m) = max\lbrace \chi(G),\chi(K_m)\rbrace = m$ since $G$ is m-colorable. And since $\chi(H)\geq \frac{n(H)}{\alpha(H)}$, $\alpha(G\square K_m) \geq \frac{n(G)m}{m} = n(G)$.

\textit{Sufficiency}

\end{proof}


% -----------------------------------------------------------------------------
\item[\textbf{7.7}]
\item[(a)]
\begin{proof}
Construct a graph $G$, area as vertex, adjacent if areas share a common edge. Because no three intersecting at a point, $\omega(G) = 2$ and no odd cycle exists; i.e. $G$ is a bipartite. We can find an orientation $D$ (from partite $X$ to $Y$) such that $ l(D) = 1$. According to \textbf{Gallai's Thm}, then $\omega (G) \leq \chi(G) \leq 1+l(D) =2$.  
\end{proof}
\item[(b)]
We can always map the vertices to a line disjointly, forming an ordering of vertex. Since no three intersects at a point, each vertex has at most 2 neighbors earlier in the ordering. Thus, $\chi(G) \leq 3$.

% ----------------------------------------------------------------------------------------------
\item[\textbf{7.8}]

%\begin{figure}[hbp!]     
%    \centering
 %   \includegraphics[width=.3\linewidth]{./figure/6_17.png}
  %  \caption{Construction of $K_{5,5}$}
   % \label{17}
%\end{figure} 

% ----------------------------------------------------------------------------------------------
\item[\textbf{7.10}]
\textit{$\chi(G_{n,k})=k+1$ if $k+1$ divides $n$ and $\chi(G_{n,k})$ does not divides $n$. }
\begin{proof}
According to the construction of the graph $G_{n,k}$, cliques of size $k+1$ are formed. Then color the graph $G_{n,k}$ with $k+1$ colors. If $k+1$ divides $n$, $1,2,3...,k+1,1,2,3...,k+1$ will be a proper coloring of graph $G_{n,k}$.

If $k+1$ does not divides $n$, let $n = q(k+1) + r = (q-r)(k+1) + r(k+2), 1\leq r < (k+1)$. Since $n\geq k(k+1)$, $q\geq k \geq r$. Therefore, a proper coloring exists that color $q-r$ times with $k+1$ colors and $r$ times with $k+2$ colors. $\chi(G_{n,k}) = k+2$.
\end{proof}


% ----------------------------------------------------------------------------------------------
\item[\textbf{7.12}]

\end{enumerate}

\textbf{Bibliography} 

\end{document}

