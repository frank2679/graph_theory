\documentclass[12pt]{article}
\usepackage{geometry}
\usepackage{amsmath}
\usepackage{amsthm}
\usepackage{graphicx}  % used to insert graph
\usepackage{subcaption}  % used to insert subgraph
\usepackage{amssymb}
\usepackage{enumitem}
\usepackage{fancyhdr}
\usepackage{tikz}
\usetikzlibrary{trees}

%% for code listing
\usepackage{listings}
\usepackage{color}
\definecolor{dkgreen}{rgb}{0,0.6,0}
\definecolor{gray}{rgb}{0.5,0.5,0.5}
\definecolor{mauve}{rgb}{0.58,0,0.82}
\lstset{frame=tb,
  language=C,
  aboveskip=3mm,
  belowskip=2mm,
  showstringspaces=false,
  columns=flexible,
  basicstyle={\small\ttfamily},
  numbers=none,
  numberstyle=\tiny\color{gray},
  keywordstyle=\color{blue},
  commentstyle=\color{dkgreen},
  stringstyle=\color{mauve},
  breaklines=false, %true,
  breakatwhitespace=false, %true,
  tabsize=3
}


\geometry{left=2.5cm,right=3 cm, top=2.5cm,bottom=2.5cm}
\lhead{Chapter 7-2}
\chead{Graph Theory HW-14}
\rhead{Yang Hang, r03942126}

\begin{document}
\pagestyle{fancy}
\begin{enumerate}
% -----------------------------------------------------------------------------
\item[\textbf{7.13}] 
\begin{proof}
\textit{All planar graph with at most 12 vertices can be 4-colored.}
We prove by induction on $n(G)$. For planar graph, we have $e\leq 3n-6$ and $\delta (G) \leq 5$. $ \sum_{v\in V} deg(v) = 2e \leq 60$. When $\delta (G) = 5$, it can only be icosahedron, which is 4-colorable.

\setlength\parindent{12pt} For $\delta(G) \leq 4$, we can always find a vertex $\lbrace v: deg(v)\leq 4\rbrace$. By induction hypothesis, the remaining graph without $v$ is 4-colorable. For $\delta(G)=4$, we can find a Kempe chain. If the chain connects two non-sequential neighbours of $v$, we can always find another Kempe chain which will not connect non-sequential neighbours of $v$. Then exchanging the colors on the chain yields a color for $v$. That's all.

\textit{Planar graph with at most 32 edges is 4-colorable.}
For planar graph, $\delta (G) \leq 5$. When $\delta (G) = 5$, $5\cdot 13>2e=64$ which implies that $n(G)\leq 12$. Apply above conclusion, it is 4-colorable.
When $\delta (G) \leq 4$, it is the same way with above to prove 4-colorable. 
\end{proof}

% -----------------------------------------------------------------------------
\item[\textbf{7.16}]
\begin{proof}
\textit{For the graph $G_n$ with even n, every proper 4-coloring of $G_n$ uses each color on exactly n vertices.}
$G_n$ is constructed by adding a new 4-cycle surrounding $G_{n-1}$. $G_2$ is as Figure \ref{7_16}. When $n$ is even, two consecutive 4-cycles will be isomorphic to $G_2$. If such $G_2$ is proved each color class covers exactly 2 vertices, adjustment could be implemented to color all consecutive $G_2$. Following is the proof of proper 4-coloring of $G_2$ using each color on exactly 2 vertices.

\setlength\parindent{12pt} It is easy to give a proper 4-coloring of $G_2$. What's more, one color class could cover at most 2 vertices since we cannot find three vertices which are dis-adjacent to each other.  Therefore, one color covers exactly 2 vertices.
\end{proof}

\begin{figure}[ht!]
\begin{minipage}{.5\textwidth}
  \centering
  \includegraphics[width=.5\linewidth]{./figure/7_16.png}
  \caption{$G_2$}
  \label{7_16}
\end{minipage}%
\begin{minipage}{.5\textwidth}
  \centering
  \includegraphics[width=.8\linewidth]{./figure/7_17.png}
  \caption{An example of Art Gallery}
  \label{7_17}
\end{minipage}
\end{figure}
%----------------------------------------------------------------------------------------------
\item[\textbf{7.17}]
\item[(a)]

\begin{proof} (With \textbf{Four Color Thm})
\setlength\parindent{12pt} \textit{Every outerplanar graph is 3-chromatic.}
Outerplanar graph is planar. So it can be 4-colorable according to \textbf{Four Color Thm}. Given a sequence of a outerplanar and using greedy coloring algorithm, only when four vertices $v_a,v_b,v_c, v_d$ are adjacent to each other, i.e., subdivision of $K_4$ as a subgraph, 4 colors are needed(\textbf{Dirac's Thm}). However, outerplanar graph is $K_4$-free. Therefore, outerplanar graph is 3-chromatic.
\end{proof}

\item[(b)]
\begin{proof} (Without \textbf{Four Color Thm})
\textit{Every outerplanar graph is 3-chromatic.}
We prove by giving a 3-coloring of outerplanar graph. Outerplanar graph has all vertices as the boundary of exterior plane. 

\setlength\parindent{12pt} If an outerplanar graph $G$ has cycles $C_i$, every two cycles share only one edge(if not, some vertex will not be the boundary of exterior plane). And see $C_i$ as vertex sequence of graph $G^\prime$, no cycle exists in $G^\prime$, i.e., a tree.

\setlength\parindent{12pt} We have known that $C_i$ is 3-chromatic(when $n(C_i)$ is odd, 3-coloring is needed). When one cycle is colored, we can always color the adjacent cycles with 3 colors until all cycles are colored. The remaining parts, which are trees, can be 2-colored.  

\setlength\parindent{12pt} If no cycle exists, $G$ will be a tree, which is 2-chromatic. Then, a 3-coloring is given. 
\end{proof}
\item[(c)]
\textit{Art Gallery Theorem}.
Given a polygon as Figure \ref{7_17}, we can add some straight lines to triangulate the polygons. 
Consider a 3-coloring of such a polygon so that each triangular is colored with 3 colors(see Figure \ref{7_17}). Then each color class has at least $\lfloor n/3 \rfloor$ elements.
In each triangular, the corner can see all points of the triangle. Hence, put a guard at the place of one color class will see the whole art gallery, i.e., $\lfloor n/3 \rfloor$ guards are needed.
%----------------------------------------------------------------------------------------------
\item[\textbf{7.19}]
\textit{ Whether exist such an edge $xy$ that $\displaystyle \min_{xy\in E} (deg(x)+deg(y))>11$. }
Add a vertex in each plane of regular icosahedron and join the vertex with all the vertices of the plane $\displaystyle \min_{xy\in E} (deg(x)+deg(y))=13$.


% ----------------------------------------------------------------------------------------------
\item[\textbf{7.20}]

\begin{proof}
\textit{Every torus-embedded graph is 7 chromatic.}
For a torus-embedded graph $G$, if we can find a region which has at most 6 adjacent regions, then induct on the $n(G)$. Given $G$, when we shrink a region to a point, then with the induction hypothesis, the remaining graph will be 7 chromatic. Add the removing region will always be colored with one of the 7 colors because it has only six adjacent regions.

\setlength\parindent{12pt} Now, we prove such a region exists. Because each region has at most 3 edges, each region will have at least 3 adjacent regions, i.e., $3n \leq 2e$. With Eular's formula on torus $n-e+f=0$, we have $e \leq 3f$. 
If every region has more than 7 adjacent regions, $7f \leq 2e$. Then $2e \geq 7f \geq 6f \geq 2e$, which is a contradiction.
\end{proof}
\end{enumerate}


\end{document}

