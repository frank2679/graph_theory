\documentclass[12pt]{article}
\usepackage{geometry}
\usepackage{amsmath}
\usepackage{amsthm}
\usepackage{graphicx}  % used to insert graph
\usepackage{subcaption}  % used to insert subgraph
\usepackage{amssymb}
\usepackage{enumitem}
\usepackage{fancyhdr}
\usepackage{tikz}
\usetikzlibrary{trees}

%% for code listing
\usepackage{listings}
\usepackage{color}
\definecolor{dkgreen}{rgb}{0,0.6,0}
\definecolor{gray}{rgb}{0.5,0.5,0.5}
\definecolor{mauve}{rgb}{0.58,0,0.82}
\lstset{frame=tb,
  language=C,
  aboveskip=3mm,
  belowskip=2mm,
  showstringspaces=false,
  columns=flexible,
  basicstyle={\small\ttfamily},
  numbers=none,
  numberstyle=\tiny\color{gray},
  keywordstyle=\color{blue},
  commentstyle=\color{dkgreen},
  stringstyle=\color{mauve},
  breaklines=false, %true,
  breakatwhitespace=false, %true,
  tabsize=3
}


\geometry{left=2.5cm,right=3 cm, top=2.5cm,bottom=2.5cm}
\lhead{Chapter 7-2}
\chead{Graph Theory HW-14}
\rhead{Yang Hang, r03942126}

\begin{document}
\pagestyle{fancy}
\begin{enumerate}
% -----------------------------------------------------------------------------
\item[\textbf{7.13}] 
\textit{$G$ is m-colorable if and only if $\alpha (G\square K_m) \geq n(G)$.}
\begin{proof} 
\textit{Necessity} 
$\chi(G\square K_m) = max\lbrace \chi(G),\chi(K_m)\rbrace = m$ since $G$ is m-colorable. And since $\chi(H)\geq \frac{n(H)}{\alpha(H)}$, $\alpha(G\square K_m) \geq \frac{n(G)m}{m} = n(G)$.

\textit{Sufficiency}
If $\alpha(G\square K_m) \geq n(G)$, $\alpha(G\square K_m) = n(G)$, because $G\square K_m$ contain $n(G)$ copies of $K_m$. $\alpha(G\square K_m)$ contains at most one vertex of each copy of $K_m$.\\
Not complete.
\end{proof}
Question: Why not does this proposition be stated as: \textit{$G$ is m-colorable if and only if $\alpha (G\square K_m) = n(G)$}? 

% -----------------------------------------------------------------------------
\item[\textbf{7.16}]
\textit{For the graph $G_n$ with even n, every proper 4-coloring of $G_n$ uses each color on exactly n vertices.}
\begin{proof}
$G_n$ is constructed by adding a new 4-cycle surrounding $G_{n-1}$. $G_2$ is as Figure \ref{7_16}. When $n$ is even, two consecutive 4-cycles will be isomorphic to $G_2$. If such $G_2$ is proved each color class covers exactly 2 vertices, adjustment could be implemented to color all consecutive $G_2$. Following is the proof of proper 4-coloring of $G_2$ using each color on exactly 2 vertices.\\
It is easy to give a proper 4-coloring of $G_2$. What's more, one color class could cover at most 2 vertices since we cannot find three vertices which are dis-adjacent to each other.  Therefore, one color covers exactly 2 vertices.
\end{proof}

\begin{figure}[ht!]
\begin{minipage}{.5\textwidth}
  \centering
  \includegraphics[width=.5\linewidth]{./figure/7_16.png}
  \caption{$G_2$}
  \label{7_16}
\end{minipage}%
\begin{minipage}{.5\textwidth}
  \centering
  \includegraphics[width=.8\linewidth]{./figure/7_17.png}
  \caption{Outerplanar graph}
  \label{7_17}
\end{minipage}
\end{figure}
%----------------------------------------------------------------------------------------------
\item[\textbf{7.17}]
\item[(a)]
\textit{Every outerplanar graph is 3-chromatic.}
\begin{proof}
Outerplanar graph is 
\end{proof}
%----------------------------------------------------------------------------------------------
\item[\textbf{7.10}]
\textit{$\chi(G_{n,k})=k+1$ if $k+1$ divides $n$ and $\chi(G_{n,k}) = k+2$ if $k+1$ does not divide $n$. }
\begin{proof}
According to the construction of the graph $G_{n,k}$, cliques of size $k+1$ are formed. Then color the graph $G_{n,k}$ with $k+1$ colors. If $k+1$ divides $n$, $1,2,3...,k+1,1,2,3...,k+1$ will be a proper coloring of graph $G_{n,k}$.

If $k+1$ does not divides $n$, let $n = q(k+1) + r = (q-r)(k+1) + r(k+2), 1\leq r < (k+1)$. Since $n\geq k(k+1)$, $q\geq k \geq r$. Therefore, a proper coloring exists that color $q-r$ times with $k+1$ colors and $r$ times with $k+2$ colors. $\chi(G_{n,k}) = k+2$.
\end{proof}


% ----------------------------------------------------------------------------------------------
\item[\textbf{7.12}]
\textit{For unit-distance graph $G$, $4\leq\chi(G) \leq 7$}
\begin{proof}
\textit{Lower bound.} From the definition of unit-distance graph, we find the $\omega(G) = 3$. We claim 3-coloring is not a proper coloring for $G$. 
Consider two equilateral triangles of side-length one sharing an edge. Two corners of the common edge are color 1 and color 2. The other two disjoint corners are color 3, since their distance is $\sqrt{3}$. 
Then consider a circle with radius $\sqrt{3}$. The vertex on the circle must be the same with the center. However, two vertices with distance 1 exist on the circle. Therefore 3-coloring is not a proper coloring. 

\textit{Upper bound.} Give an explicit 7-coloring of graph $G$ as Figure \ref{7_12}. Color 7 hexagons(six hexagons surround one, each has maximal diameter 1) with 7 colors. Move such a 7-color panel to fill the plane. 
The shortest distance between two hexagons with the same color is greater than 1. It is a proper coloring. 
\end{proof}
\end{enumerate}


\end{document}

