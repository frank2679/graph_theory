\documentclass[12pt]{article}
\usepackage{geometry}
\usepackage{amsmath}
\usepackage{amsthm}
\usepackage{graphicx}  % used to insert graph
\usepackage{subcaption}  % used to insert subgraph
\usepackage{amssymb}
\usepackage{enumitem}
\usepackage{fancyhdr}
\usepackage{tikz}
\usetikzlibrary{trees}

%% for code listing
\usepackage{listings}
\usepackage{color}
\definecolor{dkgreen}{rgb}{0,0.6,0}
\definecolor{gray}{rgb}{0.5,0.5,0.5}
\definecolor{mauve}{rgb}{0.58,0,0.82}
\lstset{frame=tb,
  language=C,
  aboveskip=3mm,
  belowskip=2mm,
  showstringspaces=false,
  columns=flexible,
  basicstyle={\small\ttfamily},
  numbers=none,
  numberstyle=\tiny\color{gray},
  keywordstyle=\color{blue},
  commentstyle=\color{dkgreen},
  stringstyle=\color{mauve},
  breaklines=false, %true,
  breakatwhitespace=false, %true,
  tabsize=3
}


\geometry{left=2.5cm,right=3 cm, top=2.5cm,bottom=2.5cm}
\lhead{Chapter 6-2}
\chead{Graph Theory HW-12}
\rhead{Yang Hang, r03942126}

\begin{document}
\pagestyle{fancy}
\begin{enumerate}
% -----------------------------------------------------------------------------
\item[\textbf{6.10}] 
 Contracting edge $(0,1),(2,3),(4,5),(5,6),(6,7), (8,9), (10,11), (11,12), (12,13)$ and deleting edge $(4,13), (5,10)$ yield a $K_5$, i.e. $K_5 \leq G$.
According to \textbf{Wagner Thm}, Headwood graph is not planar graph.
% -----------------------------------------------------------------------------
\item[\textbf{6.16}]

% ----------------------------------------------------------------------------------------------
\item[\textbf{6.17}]
According to \textbf{Property 6.26}, because $K_{4,4}$ is triangular-free graph, $t(K_{4,4}) \geq \lceil \frac{16}{12}\rceil = 2$. $K_{4,4}$ can be formed by combine two $K_{4,2}$. Hence $t(K_{4,4}) = 2$.

Similarly, $t(K_{5,5}) \geq \lceil \frac{25}{16} \rceil = 2$. Construction of two planar graph to form $K_{5,5}$ as Figure \ref{17}. 
\begin{figure}[hbp!]     
    \centering
    \includegraphics[width=.3\linewidth]{./figure/6_17.png}
    \caption{Construction of $K_{5,5}$}
    \label{17}
\end{figure} 

% ----------------------------------------------------------------------------------------------
\item[\textbf{6.19}]
\item[(a)]
\begin{proof}[$3c(K_{n,n}) \leq f(n) \leq 3 {\binom n2}^2$]
\textit{Lower Bound} Consider 3 $K_{n,n}$ in $K_{n,n,n}$ and each crossing is calculated only once. Hence the lower bound is $3c(K_{n,n})$.\\
\textit{Upper Bound} Consider $\binom 32$ selections of two partite sets and $\binom n2$ selections from each partite sets. Then yield the upper bound $3{\binom n2}^2$. 
\end{proof}
\item[(b)]
\begin{proof}

\end{proof}
\item[(c)]
\begin{proof} [$f(n) \geq n^3(n-1)/6$]
$K_{n,n,n}$ contains $n^3$ $K_{n-1,n-1,n-1}$. Crossing formed by vertices two partite sets is counted $n(n-2)^2$ times. Crossing formed by vertices from three partite sets is counted $(n-1)^2(n-2)$ times. Then each crossing is counted at most $(n-1)^2(n-2)$ times. Then $f(n) \geq \frac{n^3}{(n-1)^2(n-2)} f(n-1)$. By recurrence, $f(n) \geq n^3(n-1)f(3)/54$. And from (2), substitute $f(3) \geq 9$ to last inequation yields the result. 
\end{proof}

\item[(d)]
\begin{proof}[$f(n) \leq \frac{9}{16} + O(n^3)$]

\end{proof}

% ----------------------------------------------------------------------------------------------
\item[\textbf{6.20}]
\item[(a)]
\begin{proof} [$c(K_{m,n})\geq m\frac{m-1}{5} \lfloor\frac{n}{2}\rfloor \lfloor \frac{n-1}{2}\rfloor$. ]
$K_{m,n}$ has $\binom m6$ $K_{6,n}$.  Assume $w,x,y,z$ exists a crossing; $x,y$ are among the 6 vertices while $w,z$ are among the n vertices. The crossing appears $\binom {m-2}4$ times. \\
Hence, $\binom {m-2}4 c(K_{m,n}) \geq \binom m6 c(K_{6,n})$, which substitutes $c(K_{6,n})$ will yield the target result.
\end{proof}
\item[(b)]
\begin{proof} [$c(K_n)\geq \frac{1}{80}n^4 + O(n^3)$.]
Consider copies $K_{6,n-6}$ in $K_n$. There will be $\binom n6$ copies. As for a crossing $w,x,y,z$, it appears $(\binom 42-2)\binom {n-4}4 = 4\binom {n-4}4$ times.\\
Hence $4\binom {n-4}4 c(K_n) \geq \binom n6c(K_{6,n})$. Replacing the inequation will yield the result.
\end{proof}
\end{enumerate}

\textbf{Bibliography} 

[1] Thickness of graph, retrieve from:  mathworld.wolfram.com/GraphThickness.html
\end{document}

